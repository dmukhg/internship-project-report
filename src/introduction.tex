\chapter{Introduction}

\section{Requirements for WDAT}

The Web Data Analysis Tool, commonly referred to, in this report as
WDAT had three broad areas of functionality to deal with: 

\subsection{Browse Data}

Data on the g-node servers is stored via a common set of objects
for dealing with electrophysiological data.  It is based on the
\htmladdnormallinkfoot{NEO}{http://neuralensemble.org/trac/neo} Data
Object Model.  The Neo approach provides common names and concepts to
deal with electrophysiological data in an easy and well-structured
manner.  The NEO model is used because it is already used in major
neuroscientific projects and is also open enough to allow cooperataion
with other initiatives.

For WDAT, browsing data allowed the user to first select what kind of
container-element to browse by.  Container-elements are somewhat
similar to file-folders in a computer application, in that they
contain other container as well as data objects.  Containers are
shown in a tree-view.  Selecting a node or even a branch in this tree
shows all the non-container elements that are mapped to this container
element.  These non-container elements are called 'Data Objects' because
these are the actual experimental data which may be plotted on to a
canvas ( discussed later ).

So, for the Data Browsing component, there were three submodules
involved: the object tree, the data-object list, and a so called
selection-basket.

\subsection{View/Plot Data}

Plot Data

\subsection{Managing MetaData}

Manage metadata


