\chapter{Introduction}

\section{Requirements for WDAT}

The Web Data Analysis Tool, commonly referred to, in this report as
WDAT had three broad areas of functionality to deal with: 

\subsection{Browse Data}

Data on the g-node servers is stored via a common set of objects
for dealing with electrophysiological data.  It is based on the
\htmladdnormallinkfoot{NEO}{http://neuralensemble.org/trac/neo} Data
Object Model.  The Neo approach provides common names and concepts to
deal with electrophysiological data in an easy and well-structured
manner.  The NEO model is used because it is already used in major
neuroscientific projects and is also open enough to allow cooperataion
with other initiatives.

For WDAT, browsing data allowed the user to first select what kind of
container-element to browse by.  Container-elements are somewhat
similar to file-folders in a computer application, in that they
contain other container as well as data objects.  Containers are
shown in a tree-view.  Selecting a node or even a branch in this tree
shows all the non-container elements that are mapped to this container
element.  These non-container elements are called 'Data Objects' because
these are the actual experimental data which may be plotted on to a
canvas ( discussed later ).

So, for the Data Browsing component, there were three submodules
involved: the object tree, the data-object list, and a so called
selection-basket.

\subsection{View/Plot Data}

The data-objects were among a set of plottable-types.  By plottable, I
mean the consisted of a set of numeric, time-series data that could be
represented in a graphical manner.  Once a selection had been made in
the Browse Data panel, a  second panel could be invoked that would
serve to contain the plotted details of all the data-objects.  This
"View" panel would also be broken up into multiple parts.

A legend module in the View panel would contain a representation of
the selection along with unique color-codes for the individual
data-objects (for easy identifiability in the plots).

Also, there would be three separate modules showing different zoomed
views of the data.  The one containing the entire width of the
time-domain is called total.  Selections made on the 'total' panel
would be shown on a second zoom level called 'zoom'.  Selections made
on the 'zoom' panel would be shown on a third zoom level called
'detail'.  This three tier design was thought to be most easy to
maneuvre through the entire time-series.

\subsection{Managing MetaData}

Simply storing data derived from the experiment is not enough.  One
also requires to store the conditions of the experiment in reasonable
detail to be able to reproduce the results at a later point in time.
This is where MetaData steps in.  MetaData on the G-Node servers is
maintained via an API derived from the odMl (open metadata Markup
    Language) specification available at
\htmladdnormallinkfoot{www.g-node.org}{http://www.g-node.org/projects/odml}.

For the metadata panel, it was required that there be convenient and
non-obstrusive means to view and edit metadata.  Also, there would be
easy means to connect metadata to selections made in the browse-panel.


\section{Implementation and delineation of the panels}

There was a clear purpose to each of the various panels vis-a-vis
data-explorer, data-visualization and metadata-management.  After some
discussion, it was decided that there is no overlap between these
panels and hence, it is best to separate them visually via tabs.  Tabs
were chosen because they reflect the state of the application and
allow the user to switch beween interfaces naturally.  Since there is
no way to know which of the three panels the user may wish to view at
any time, a wizard-like sequence of panels was out of the question.

The tabs were individually labelled 'explorer', 'plot' and 'metadata'.

