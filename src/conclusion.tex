\chapter{Conclusion}

The Web Data Analysis Tool, in its current implementation stage has a
lot of features missing.  Most notable among them being the lack of
search functionality and the general clumsyness of the UI and code.

However, a lot has been acheived in the small time duration of the
internship.  Among the chief tasks that lie ahead of us, none is as
important as refactoring source code into a more pliable and
consistent state.  This will help maintain the code in the long run.
Also, there is a need to revamp the UI to modern standards.  A more
thoughtfully polished user-interface that needs lesser clicks on the
part of the end-user would be a welcome change.  Support for faster
metadata management also needs to be taken care of.  Also, the core
functionality of choosing signals to either view or annotate should be
kept in mind when the next round of development is done.


\section{Acknowledgements}

I would like to thank Thomas Wachtler for offering me this internship
in the first place.  His guidance and expertise in neuroscience helped
a lot in assessing the goals of this developmental enterprise.

Also, I would like to thank Andrey Sobolev, my mentor during the
course of this internship without whose well-meaning criticism, the
project would not be in the advanced state that it is in.

Also, towards the end, Adrian Stoewer's comments on the modularity of
the source code helped a lot in chalking out a plan of development for
the next few rounds. 

I learnt a lot during my work with the German Neuroinformatics Node. I
am assured, it made me a better developer and made me realise the
power of the end user in deciding the functionality of any
application.
