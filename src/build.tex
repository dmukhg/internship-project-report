\chapter{The Build Processes}

The build process is mostly automated and utilizes the GNU make build
tool. A cursory glance at the Makefile located in the 'wdat' directory
will tell you all that there is to know but is also documented in this
report for completeness's sake.

\section{Variables}

\begin{description}

  \item[JS\_COMPILER] \hfill \\
  This defaults to uglify-js which is a node-js tool that minifies javascript files.

  \item[WIDGET\_FILES] \hfill \\ 
  This a space-separated-list of all files that are written as widgets
  and are therefore to be in memory before they are used.  (This
  sequential loading of components is managed by the build process
  as will be examined later)

  \item[SUBMODULE\_FILES] \hfill \\
  A similar space-separated-list of all files that are submodules and
  are invoked within the definitions of modules.  Submodules, upon
  invocation, return objects that are used within modules to operate
  upon them. 

  \item[JS\_FILES] \hfill \\
  This is the main compilation unit.  The order of files and variables
  within this variable is what infulences loading order within the
  browser.

  \item[CSS\_FILES] \hfill \\
  This the compilation unit for css files. The loading order for css
  is not really as important as the javascript equivalent. But the
  developer must still be aware of cascade rules in css such that
  results are exactly as expected.
\end{description}


\section{Build Process}

\subsection{Concatenation}

The compilation units are individually concatenated, while preserving
file-order, into single files. The concatenated CSS file is processed
further.  The concatenated file stays as 'wdat.prep.css'.

\subsection{Minification}

For smaller footprint, the generated js file 'wdat.js' is minified
using
\htmladdnormallinkfoot{UglifyJS}{http://github.com/mishoo/uglifyjs}.
As of now, there is no minification for the css file.

\subsection{Compilation}

The Cascading Style Sheets use the
\htmladdnormallinkfoot{LESS}{http://lesscss.org/} extensions for css
and hence must be compiled to browser compatible css before proceeding
further.  This compiled unit is called 'wdat.css'.

\section{Usage}

Make usage follows common conventions prevalent in the open-source
domain.

\subsection{Building non-minified scripts}

\lstset{language=sh, tabsize=4}
  \begin{tikzpicture}
    \node [fill=codebgcolor, rounded corners=5pt]
{{
  \begin{tabular}{l}
  \begin{lstlisting}
  make
  \end{lstlisting}
  \end{tabular}
}};
\end{tikzpicture}

This creates the wdat.js and wdat.css files.  Note, this doesn't cause
any minification.

\subsection{Building minified scripts}

\lstset{language=sh, tabsize=4}
  \begin{tikzpicture}
    \node [fill=codebgcolor, rounded corners=5pt]
{{
  \begin{tabular}{l}
  \begin{lstlisting}
  make wdat.min.js
  \end{lstlisting}
  \end{tabular}
}};
\end{tikzpicture}

This builds wdat.min.js only.  It first creates the wdat.js file and
then passes it through UglifyJS to create the minified javascript
file.  A minified javascript file is perfectly fine to execute on a
browser.  It uses the same global namespace as the original js file.
Only, unneeded characters like carriage-return and spaces are removed
and the functions renamed to shorter names.

\subsection{Building the CSS files}

\lstset{language=sh, tabsize=4}
  \begin{tikzpicture}
    \node [fill=codebgcolor, rounded corners=5pt]
{{
  \begin{tabular}{l}
  \begin{lstlisting}
  make wdat.css
  \end{lstlisting}
  \end{tabular}
}};
\end{tikzpicture}

This creates the less-compiled version of the wdat.css styles.


\section{Additional notes}

The make process doesn't require user input and can be invoked
programmatically or via hooks (in the case of deployments).  This
makes the application capable of building itself when a new release is
pushed onto the servers.
